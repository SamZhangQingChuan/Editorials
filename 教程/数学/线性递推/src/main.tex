\documentclass{article}
\usepackage{graphicx}
\usepackage{amsfonts}
\usepackage[utf8]{inputenc}
\usepackage{xeCJK}
\usepackage{amsmath}
\usepackage{amsthm}
\usepackage{bm}
\usepackage{minted}
\usepackage{hyperref}
\usepackage[dvipsnames]{xcolor}
\usepackage{parskip}
\usepackage{soul}
\usepackage{amssymb}
\usepackage{cancel}
\usepackage{ragged2e}
\usepackage[normalem]{ulem}
\usepackage[stable]{footmisc}
\newtheorem*{lemma*}{Lemma}


\title{线性递推\footnote{更多内容请访问:\url{https://github.com/SamZhangQingChuan/Editorials}}}
\author{张晴川\\\href{mailto:qzha536@aucklanduni.ac.nz}{\texttt{qzha536@aucklanduni.ac.nz}}}

\begin{document}
\maketitle

\section*{Random Number Generator \footnote{\url{https://www.codechef.com/problems/RNG}}}
\subsection*{大意}

给定线性递推数列初值:$A_1,A_2,\ldots ,A_K$,转移关系 $A_i = \sum_{j=1}^K C_j A_{i-j}$。

求 $A_N \bmod 104857601$(NTT 模数)。

\subsection*{数据范围}
\begin{itemize}
\item $ 1 \le N \le 10^{18}$
\item $0 \le A_i, C_i < 104857601$
\end{itemize}
\subsection*{题解}
设 $v_0 = [A_1,A_2,\ldots ,A_K]^T$

由于是线性递推,必定存在一个转移矩阵 $M$,满足:
$$
M^n v_0 = [A_{n+1},A_{n+2},\ldots ,A_{n+K}]^T
$$

于是我们的目标就是求 $M^{N-1}$ 的最低项。

按定义可以得到:
$$
M^K v_0 = [A_{K+1},A_{K+2},\ldots ,A_{K+K}]^T
$$

考虑右边的最低项 $A_{K+1}$ ,根据递推关系,我们有:
\begin{align*}
A_{K+1} 
&= \sum_{i=1}^K C_1 A_{K+1-i}\\
&= C_1 A_K + C_2 A_{K-1} +\cdots + C_K A_1
\end{align*}
同理可得 $A_{K+2},A_{K+3},\ldots ,A_{K+K}$ 也有类似关系。

于是我们得到了:
$$
M^K v_0= C_1 M^{K-1}v_0 + C_2 M^{K-2} v_0 + \cdots + C_K M^{0} v_0
$$
即:
$$
(M^K - C_1 M^{K-1}  - C_2 M^{K-2}  - \cdots - C_K M^{0}) v_0 = 0
$$

由于 $v_0$ 其实可以是任意向量,$M^K - C_1 M^{K-1}  - C_2 M^{K-2}  - \cdots - C_K M^{0}$ 其实就是零矩阵。

设 $f(x) = x^{N-1},g(x) = x^K - C_1 x^{K-1}  - C_2 x^{K-2}  - \cdots - C_K x^{0}$

那么必然存在商$q(x)$和余数$r(x)$ 满足 $f(x) = q(x)g(x) + r(x)$,其中 $r$ 的次数低于 $g$ 的次数。

由于 $f(M) = q(M)g(M) + r(M) = q(M)0 + r(M) = r(M)$,我们只需计算 $r(x)$。这可以通过快速幂计算,中间过程可以暴力平方取模,或者使用多项式的技巧取模,复杂度分别为 $O(\log(N)K^2)$ 和 $O(\log(N)K\log(K))$。

假设计算完之后,得到 $r(x) = \sum_{i=0}^{K-1}c_{i} x^{i}$,那么 $f(M) = \sum_{i=0}^{K-1}c_{i} M^{i}$。在 $0 \le i < K$ 的时候,$M^i v_0$ 的最低项其实就是 初值 $A_{i+1}$。于是最终答案就是 $\sum_{i=0}^{K-1}c_{i} A_{i+1}$。

\subsection*{复杂度}
\begin{itemize}
\item 时间:$O(\log(N)K\log(K))$
\item 空间:$O(K)$
\end{itemize}
\subsection*{代码}
\url{https://gist.github.com/42403551c8080416af4b8f2baeaf5016}

\end{document}
