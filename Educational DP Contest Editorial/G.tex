\section{Longest Path}
\subsection*{题意}
给定一个 $n$ 个点,$m$ 条边的\textbf{有向无环图},求图中最长路径的长度。定义路径长度为边的数量。
\subsection*{数据范围}
\begin{itemize}
\item $2 \leq N \leq 10^5$
\item $1 \leq M \leq 10^5$

\end{itemize}

\subsection*{题解}

本题是\textbf{所有}动态规划模型的\textbf{理论}基础。抽象地说,所有动态规划的本质都是在一张(带权)有向无环图上求最长路,之前做的\textit{背包},\textit{最长公共子序列} 皆是如此。对应关系如下:
\begin{center}
\begin{tabular}{ |c|c| } 
 \hline
 \textbf{动态规划} & \textbf{有向无环图} \\
 \hline
 状态 & 节点\\ 
 \hline
 转移方程 & 边\\ 
 \hline
 无后效性 & 图中无环\\ 
 \hline
 $\cdots$ & $\cdots$\\
 \hline
\end{tabular}
\end{center}

这题的转移方程比较显然,即 ${\texttt{dp}[i]}$ 表示以节点 $i$ 为终点的路径中最长的长度。转移方程为:
$$
{\texttt{dp}[i]} = \max\{{\texttt{dp}[j]} + 1 \mid \text{存在边 }\ j\rightarrow i\}
$$

有向无环图最重要的性质就是没有环,这使得我们可以对节点进行拓扑排序,并按照拓扑序的顺序计算 {\texttt{dp}} 状态的值。相信你对于宽度优先搜索({\texttt{BFS}})没有什么问题,以下我给出一种用深度优先搜索({\texttt{DFS}})的做法,又称记忆化搜索。

\subsection*{核心代码}
\inputminted[linenos,autogobble]{cpp}{./Code/G.cpp}
\newpage