\section{Grouping}
\subsection*{题意}
有 $n$ 只兔子,现在把兔子分为若干组,如果兔子 $i$ 和兔子 $j$ 在同一组,收益就加上 $ a_{i,j}$。求最大收益。
\subsection*{数据范围}
\begin{itemize}
\item $1 \leq n \leq 16$
\end{itemize}


\subsection*{题解}

首先设 $\texttt{cost}[\texttt{mask}]$ 表示 $\texttt{mask}$ 对应的子集划分在一组的收益。这一步计算非常简单,用暴力的 $O(2^n\cdot n^2)$ 方法即可:
\inputminted[linenos,autogobble]{cpp}{./Code/U2.cpp}

接下来设 $\texttt{dp}[\texttt{mask}]$ 表示将$\texttt{mask}$分成若干组的最大收益。计算方法也很简单,对于 $\texttt{mask}$,只需要枚举子集 $\texttt{submask}$ 递归计算:
$$
\texttt{dp}[\texttt{mask}] = \max_{\texttt{submask} \subseteq \texttt{mask}} (\texttt{cost}[\texttt{submask}] + \texttt{dp}[\texttt{mask} - \texttt{submask}])
$$

但朴素的方法会导致复杂度高达 $O(2^n \cdot 2^n) = O(4^n)$。

我们考虑用如下方式枚举子集的子集:
\inputminted[linenos,autogobble]{cpp}{./Code/U3.cpp}

不难发现,\mintinline{cpp}{submask = (submask - 1) & mask} 的本质是把当前 $\texttt{submask}$ 的最后一个 $1$ 去掉,并恢复之前被去掉的 $1$。举个例子:
\begin{align*}
    \texttt{mask} &= 10101\\
    \texttt{submask} &= 10100\\
    \texttt{(submask - 1) \& mask} &= 10001\\
\end{align*}

,其中第$3$位被去掉,第$5$位被加上。

因为这种枚举方法每进行一步一定会生成一个新的子集,效率达到最优。我们现在只需要计算复杂度是否可以接受即可。

注意到每个大小为 $i$ 的子集都包含 $2^i$ 个子集,而一共有$\displaystyle\binom{n}{i}$ 个大小为 $i$ 的子集。所以根据二项式定理,总运算次数应等于:
$$
\sum_{i = 0}^n \binom{n}{i}2^i = \sum_{i = 0}^n \binom{n}{i}2^i1^{n-i} = (2+1)^n = 3^n
$$
于是可以通过。

\newpage